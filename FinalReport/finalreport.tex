% this is a comment in latex
% substitute this documentclass definition for uncommented one
% to switch between single and double column mode
\documentclass[11pt,twocolumn]{article}
%\documentclass[11pt]{article}

\usepackage{fullpage}
\usepackage{subfigure,indentfirst}
% for url
\usepackage{hyperref}
% for underlined text
\usepackage[normalem]{ulem}
% for including pdf figures
\usepackage{graphicx}

% my own versions: my_enumerate and my_itemize
\newenvironment{my_enumerate}{
  \begin{enumerate}
    \setlength{\itemsep}{1pt}
      \setlength{\parskip}{0pt}
\setlength{\parsep}{0pt}}{\end{enumerate}
}

\newenvironment{my_itemize}{
  \begin{itemize}
    \setlength{\itemsep}{1pt}
      \setlength{\parskip}{0pt}
\setlength{\parsep}{0pt}}{\end{itemize}
}

% this starts the document
\begin{document}

\title{CS87 Project Report: 
A Concise and Meaningful Paper Title}

\author{Author1, Author2, Author3 \\
Computer Science Department, Swarthmore College, Swarthmore, PA  19081}

\maketitle

\begin{abstract}

The abstract is a brief summary of your work. It should be written to make the
reader want to read the rest of your paper. Briefly state the basic contents
and conclusions of your paper: the problem you are solving, why the reader
should care about this problem, your unique solution and/or implementation,
and the main results and and contributions of your work.

  For writing organization resources, see my
  CS Research and Writing Guide:

\noindent {\small \url{www.cs.swarthmore.edu/~newhall/thesisguide} } 

And, off my help pages~\cite{newhall:help} are links to other resources 
for technical writing, including a writing style guide and information
about Unix tools for creating and editing figures.  

\end{abstract}


\section {Introduction} 

Introduction 


The introduction is the big picture of your work: what, why, and how. It
includes a definition of the problem you are solving, a high-level description
of your solution including any novel techniques and results you provide, and a
summary of the main results of your paper. In addition, motivates the problem
you are solving (why should a reader find your work important), and describes
your contribution to the area (this may not be applicable to your project).
The first paragraph of the introduction should contain all of this information
in a very high-level. Subsequent paragraphs should discuss in more detail the
problem you are solving, your solution, and your results and conclusions.


\section {Related Work}\label{relwork}
This is an essential part of a research paper; discussing related work is a
good way to put your work in context with other similar work, and to provide a
way for you to compare/ contrast your work to other's work.  You should use
feedback on the annotated bibliography from your project proposal to structure
this section; it should be written as a re-write of your annotated bibliography
in a single Related Work section.


\section {One or more sections describing your Solution}\label{soln}
Details of the problem you are solving Details of your solution and the
project's implementation Even though you may have spent an enormous amount of
time writing code, this should not include a listing of any code you wrote.
Only if your project is about developing an algorithm or a new language, may
code examples be appropriate here.  Discussion of how your solution solves the
problem.


\section {Results}\label{results}
Experimental Results demonstrating/proving your solution Explain the tests you
performed (and why) Explain how you gathered the data and details of how your
experiments were run (any system/environment set up) Present your results
Choose quality over quantity; the reader will not be impressed with pages and
pages of graphs and tables, instead s/he wants to be convinced that your
results show something interesting and that your experiments support your
conclusions.  Discuss your results!  Explain/interpret your results (possibly
compare your results to related work). Do not just present data and leave it up
to the reader to infer what the data show and why they are interesting.  

\section{Conclusions and Future Directions}\label{conc} 
Conclusions and Future Directions for your work Conclude with the main ideas
and results of your work. Discuss ways in which your project could be
extended...what's next? what are the interesting problems and questions that
resulted from your work?

\section{Meta-discussion}\label{meta} 
A brief meta-discussion of your project Include two paragraphs in this section:
Discussion of what you found to be the most difficult and least difficult parts
of your project.  In what ways did your implementation vary from your proposal
and why?  


At the end of your paper is a Reference section. You must cite each paper that
you have referenced...your work is related to some prior work.

% The References section is auto generated by specifying the .bib file
% containing bibtex entries, and the style I want to use (plain)
% compiling with latex, bibtex, latex, latex, will populate this
% section with all references from the .bib file that I cite in this paper
% and will set the citations in the prose to the numbered entry here
\bibliography{finalreport}
\bibliographystyle{plain}

% force a page break
\newpage 
% I want the Appendices to be single column pages
\onecolumn
\section{Appendices}\label{appx} 

You may optionally add appendices to your report.  They do not count towards
page total.  Appendicies are for expanding on details or results that are
beyond inclusion in the main part of paper.  It could be where you have code
snippets that illustrate some of details of what you discuss, but it is not a
venue for a dump of all the code you wrote, which has no place in research
paper.  


\newpage
\twocolumn
\section{some latex examples}

In my latex examples:
\begin{verbatim}
/home/newhall/public/latex_examples/paper/
\end{verbatim}
Is an example report and bibtex for a paper with lots of examples
of latex formatting for tables, figures, references. etc.  See 
the {\tt paper.tex} file and the {\tt paper.bib } file for lots of
examples. Also, off my   help pages is some information for how to 
convert documents from one form to another: 
{\small \url{www.swarthmore.ed/~newhall/unixlinks.html#doc} }

Here are a few latex examples:

\subsubsection{Examples of figure and section references}

To refer to a figure or section with a label, 
use {\tt backslash ref\{labelname\}} and note where the labelname is 
defined in the figure.  For example: (see Figure~\ref{dyncompex}, see 
Section~\ref{meta}). 

\subsubsection{Examples of incorporating figures}

Here is an example of how to incorporate a figure in a document...you need
to include a graphics package at the top of the latex document.  

\begin{figure}[t]
\centerline{\includegraphics[height=1.0in]{/home/newhall/public/latex_examples/report/ERbook.pdf}}

\caption{ {\label{dyncompex} {\tt Figures should have a caption that 
    describes what the figure is/shows}.  For example: During a dynamically 
    compiled execution, methods may be interpreted by the VM and/or 
    compiled into native code and directly executed. 
    {\em The native code may still interact with the VM.  In this 
      case, the VM acts like a run-time library to the AP.
}}}
\end{figure}

{\bf For all figures in the document, you should refer to it and 
describe what it shows in the prose of your paper}.  Here is an example: 
Figure~\ref{dyncompex} shows the two execution modes of an 
environment that uses dynamic compilation: (1) the VM interprets AP 
byte-codes; (2) native code versions of AP methods, that the VM compiles 
at run-time, are directly executed by the operating system/architecture 
platform with some residual VM interaction (for example, activities like 
object creation, thread 
synchronization, exception handling, garbage collection, and calls from native 
code to byte-code methods may require VM interaction). 

\subsubsection {Figure Formating}

I can force a figure to the top or bottom of a page by using the [t] or [b].
I can also try to force it to come after the prose it follows in the .tex
file by using [!htb] (latex doesn't always comply):

\subsection {column spanning figures or tables}

In two-column documents, you often have figures or tables that are
too large to fit in a single column.  In this case you can specify
that they fill both columns of the document by defining a table or 
column with a {\tt *} at the end: {\tt figure*} or {\tt table*}. Latex
will only place these at the very top or very bottom of a two-column
document.   See Table~\ref{swapdevresults} for one example.  You can
uncomment the following figure definition to generate another example:

\begin{table*}
\begin{center}
\begin{tabular}{|l||r l|r|r|r|}
\hline
\multicolumn{1}{|c||}{Benchmark} & \multicolumn{2}{c|}{Nswap2L} &\multicolumn{1}{c|}{Nswap} &\multicolumn{1}{c|}{Flash} & \multicolumn{1}{c|}{Disk} \\
\hline
WL1   & {\bf 443.0} & (3.5 speedup)    & 471.8  & 574.2  & 1547.4 \\
WL2   & {\bf 591.6} &(30.0)            & 609.7  & 883.1  & 17754.8\\
WL4   & {\bf 578.9} &(30.9)            & 591.7  & 978.4  & 17881.2\\
Radix & {\bf 110.7} &(2.3)             & 113.7  & 147.4  & 255.5 \\
HPL   & 536.1 &(1.5)             & {\bf 529.7}  & 598.7  & 815.3 \\
\hline
\end{tabular}

\caption{\label{swapdevresults} Comparison of different swap devices. 
  {\em For each benchmark, the total run time (in seconds) when 
    run using Nswap2L, Nswap Network RAM, flash or Disk as the 
    swap partition. Bold entries show the best time.  Nswap2L speedups 
over disk are in parentheses.} }
\end{center}
\end{table*}

The first part of a tabular definition {\tt e.g. |c|l|r|rr|} 
specifies the number of columns, the text alignment in each column,
(centered, left, right), and any vertical bars you want drawn between 
columns.  In this example, there are 3 columns, the first is 
centered, the others right-aligned, and there is a double 
vertical bar between the first and second columns and a single bar 
between the second and third.

Each row of values is listed on a separate line with.  Ampersands are 
used between each column's value, and backslash-backslash ends a row.   
{\tt hline} can be used to draw horizontal lines.  

The second table definition is a larger example that 
demonstrates the multicolumn directive
which can be used to change the default formating of a particular row,
including allowing content to span multiple columns, to be aligned 
differently, and to have different vertical bars drawn then the 
default column definitions listed in the tabular definition:



\end{document}

